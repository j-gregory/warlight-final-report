\documentclass[a4paper,11pt]{article}

\usepackage[affil-it]{authblk}
%\usepackage{amsfonts, amsmath, amssymb, bm}
%\usepackage{float}
\usepackage[margin=1.0in]{geometry}
\usepackage{graphicx}
%\usepackage{hyperref}
%\usepackage{subcaption}
%\usepackage{subfigure}
\usepackage{url}

%\graphicspath{{../pdf/}{../jpeg/}}
% \DeclareGraphicsExtensions{.pdf,.jpeg,.png}
\graphicspath{{figures/}}

%\newcommand{\bfnull}{\textbf{null}}

\begin{document}

\title{Towards the Planning of World Domination}
\author{Jason Gregory, Moshe Katz, \& Kamruzzaman Quddus \\ \{jgregory, mmkatz, kquddus\}@umd.edu}
\affil{University of Maryland, College Park, MD}
\date{\today}

\maketitle

%
\abstract{In this work, we propose to develop a probabilistic planner that will compete in the Warlight AI Challenge 2.  More specifically, we intend on implementing and modifying the \emph{Upper Confidence Bounds on Trees} (UCT) algorithm for winning Warlight games when randomly paired up against users with other strategies.}

\section{Introduction}\label{sec:intro}
\subsection{AI in Games}\label{aiingames}
Although there are many major commercial fields that benefit from improvements
in Artificial Intelligence (AI), one of the largest research areas in the field is
the application to games.  From the 1951 debut of artificially intelligent bots
for Nim \cite{nim}, checkers, and chess \cite{checkerschess}, to Deep Blue's 1997
defeat of chess grandmaster Garry Kasparov, and continuing to today's sophisticated
first-person-shooter adversaries, the impact and importance of AI in gaming as a
tool for understanding and furthering AI techniques for application to real-world 
problems cannot be overstated.

Despite all of this work, there still remains many "open problems" in game AI
design.  Many games still use AI bots that are quite naive, and some games have
not yet been determined to be winnable by a computer player. To encourage continued
research and development of game AI, some game makers occasionally run competitions
for developing new AI bots to play their games.

In this work, we will develop a bot to compete in the Warlight AI Challenge 2 \cite{warlight}.
The goal of this competition is to build a bot that can play (and, of course, can
win) Warlight, a RISK-style game.

\subsection{Problem Statement}\label{sec:problem}
We expect to develop an AI planner specifically for competing in the Warlight AI Challenge 2.  
To aid those who are less familiar with RISK, we first describe the game-play mechanics,
\footnote{While a full description of the rules of RISK would be too long to 
include here, the basics described should be enough to understand the general 
game-play.  The reader is directed to the printed rules document found in the game 
box for additional details.} then the differences of Warlight.

\subsubsection{RISK}
At the start of a game, players take turns placing troops on the board to claim
their territories.  After all territories have been claimed, players place
additional troops on the board, with the number of additional troops depending on
the number of game players.  All players choose the same number of territories.
When only two players are playing, as is the case in the games our bot will play,
one third of the territories remain neutral, which means that they will only defend
against attacks from the two players, but they will not attack on their own.

At the beginning of every turn, players receive additional armies to place on the 
game board.  The number of armies varies based on how many territories the player 
controls, and includes additional bonus armies if the player controls entire 
continents or if the player plays any "RISK Cards".  Players place these armies 
before continuing their turns.  The rest of a player's turn consists of attacking 
neighboring territories held by opposing troops and/or transferring troops to a 
neighboring territory held by the player's own troops in order to fortify it. 
Note, only one fortification move may be made per turn, and it must be the last 
move of the turn.

When a player attacks another player's territory, the attacker rolls a number of 
dice, as does the defender.  The dice are compared and the loser of each comparison 
removes an army token from the territory.  The attacker conquers the territory if 
the defender's last army is removed from it, in which case the attacker must move 
some of his troops into the attacked territory.  If the attacker is successful in 
their attack , they draw a "RISK Card" which can be traded-in as sets of three 
to receive additional troops in a future turn.

\subsubsection{Warlight} \label{sec:warlight}
Warlight has several modifications from the RISK game-play.  The most significant 
change is the addition of \emph{fog of war} -- that is, players can only see 
their own territories and the territories immediately adjacent to them.  
Additionally, while RISK is played on a world map that has been divided into arbitrary 
territories, Warlight can be played on custom-designed or randomly-generated maps.  
Warlight also has many additional types of cards that can be drawn, including 
Reinforcement cards (like RISK), Blockade cards, Spy and Surveillance cards, 
Gift cards, and several other types.

A major factor in developing a strategy for playing Warlight is that a multi-player 
(human) Warlight game may have many configurable options set by the game host - 
this makes developing a strategy that covers all game options impossible. Because 
RISK is usually played on a physical game board, the per-game options are limited 
to simple "house rules" modifications; for this reason, there are many published 
RISK strategy guides but very few guides for Warlight.  Furthermore, guides for 
Warlight are usually much more vague about specific implementation of the strategies 
discussed therein \cite{warlightguide}.

\subsubsection{Warlight AI Challenge 2}
For the Warlight AI Challenge 2, our bot will play a two-player game against 
another bot.  The game will be played on a random map, using the standard scoring 
and troop allocation rules, but not using any cards.

One major game setting is called the \emph{luck factor}. The luck factor is used 
in the calculation of how many attacking and defending armies are destroyed.  While 
the previous Warlight AI Challenge used a luck factor of $100\%$, this challenge uses 
only $16\%$ so there is a lot less luck and a lot more certainty about the outcome of 
an attack.

A bot wins the game by destroying all of the armies of the opposing bot.  However, 
if the turn limit of the game is reached, the game is considered a draw.  The 
maximum number of turns allowed in a game depends on the size of the map, and is 
currently set as the number of map regions times 2.5.

\subsection{Previous Work}\label{sec:previous}
The largest body of related work is the set of other bots in the current challenge.  
There are currently around 200 competition entries from 30 countries written in 
eleven languages \cite{warlight}.  While we have not yet examined any of our competitors 
in detail, they are all working toward the same goal as we are of developing an AI bot for 
this game, and many of them are much farther ahead at this time.

Additional related work comes from the first Warlight AI Challenge.  While the 
rules of the game for that competition were slightly different, many of the 
techniques used there will still apply to this challenge, either as examples of 
successful approaches, or otherwise as examples of what not to try.

There are also many strategy guides that exist for RISK.  However, as discussed 
above, there is no real complete strategy guide for Warlight.  Given that every 
game our bot will be playing will be on a completely random map, and paired against a
random opponent, many of the typical strategy guides might not apply to every game; 
this is especially true for the static RISK guide.  Additionally, many common RISK 
strategies require knowledge of more of the board status than Warlight allows us to 
have due to fog of war.

Finally, there are existing AI bots for the original RISK game.  While some have been 
discussed only theoretically, including strategies from MIT \cite{riskmit}, Markov 
Chains from North Carolina State University \cite{riskncs}, Monte-Carlo (UCT) 
techniques for territory selection from University of Alberta \cite{riskalb}, and 
studies of dice-rolls from Elon University \cite{riskelon}, implementations of AI 
players for RISK have been created and sold in commercial software as early as 1989. 
These solutions are also of varying quality - the 1992 release of \textit{WinRisk} for 
Windows 3.1 was capable of beating at-best an inexpert child player - and most 
implementations are closed-source without any documentation of their algorithms.
%there is still prior work for creating an AI bot that can play RISK and similarly-structured games.

\subsection{Importance and Relevancy}\label{sec:importance}
In addition to the obvious goal of performing well in the Warlight AI Challenge 2, 
we will use this project as a mechanism to learn and practice topics from class and 
demonstrate our ability to apply different techniques to a real-world use-case.
Moreover, techniques used for tactical game-play often have grounded 
application for training exercises and simulations of realistic military situations.  This is 
especially true for algorithms that model adversarial behavior in addition to our 
own, providing the ability to play out WarGames-style dangerous scenarios in 
complete safety.\footnote{In fact, the NORAD missile command center was already 
using the RISK-style AI shown in WarGames in 1979, and there really was an incident
in which the AI simulation was thought to be a real Soviet attack
\cite{wargamesreal}.}

%
\section{Technical Approach}\label{sec:approach}
Our approach for designing a planner for the Warlight AI Challenge 2 is to implement the basic \emph{Upper Confidence Bounds on Trees} (UCT) algorithm~\cite{uct} and then, time permitted, improve the performance of the algorithm by making different modifications.


\subsection{Hypothesis}\label{sec:hypothesis}
We believe that the Monte-Carlo-based UCT algorithm will enable the search of different applicable moves in the fast-paced, time-limited game of the Warlight AI Challenge 2. We also hypothesize that, due to the strict time constraint enforced by the game engine, we will be required to modify the UCT algorithm to expedite the search through the game tree.  By making these modifications, we feel that we can make intelligent decisions in each turn that eventually lead to winning the game.

\subsection{Process}\label{sec:process}
The general strategy for the planning algorithm is to acquire regions in a manner by which power is gained quickly while always remaining tactically flexible so as to starve the opponent from in-game advantages. Such a planner must be adversarial in nature, robust enough to adapt to different counter strategies, efficient enough to plan quickly, and adhere to the constraints defined by the games rules.

The various constraints in the Warlight AI Challenge 2 primarily dictate how we will approach developing a planner. First, the game provides each player with a $10$ second time bank that ultimately limits one from planning extensively in each turn.  Additionally, this game increases the difficulty of the original RISK game by introducing the concept of fog of war, as described in Section~\ref{sec:warlight}. This incorporates a level of uncertainty into the game that must be met with a probabilistic planner instead of a deterministic one. As a result of these constraints imposed by the game engine, we desire an algorithm that balances planning performance with run time; otherwise, we will run out of time, never make a move, and lose the game. 

From our research, classical planning techniques that exhaustively explore the entire state space are not applicable to this problem because of the time required to find an optimal plan. Instead, we propose to use the Monte-Carlo-based UCT algorithm.  First, we plan on implementing the basic UCT algorithm so that we can generate a sensible plan each turn for a single region in our possession.  Next, we plan to extend this implementation so that we will execute a new instance of the UCT algorithm for every region that contains more than one of our troops.  In other words, regions that we own and have the ability to attack or transfer will have separate game trees. We will simulate the game for each game tree simultaneously using threads, initially disregarding any time constraint, and produce a set of the best moves, each with some probability $p_{i}$. Note, due to the nature of the UCT algorithm, we will not produce an optimal solution, but rather a solution with a certain probability of success, which we can apply some metric to determine the quality of the plan of actions. We believe we are able to evaluate each region independently because the attack and transfer commands in a game are independent. For example, it is illegal to transfer troops from region $A$ to region $B$, and then use those newly-acquired troops to attack region $C$. We plan to initially evaluate the UCT algorithm without enforcing a time constraint so that we can characterize the performance of the algorithm completely and determine an upper bound for the run time required to produce all possible commands. 

Depending on the amount of time required to traverse the entire game tree in a single instance of the UCT algorithm, we may be required to implement some methods for speeding up the search. One possible method we are considering is the concept of a zone for the purposes of more efficient simulation generation.  We define a zone to be a collection of regions where each region is a neighbor to at least one region with the same owner and at least one region with an opposing owner.  From this definition we can develop a strategy for quickly identifying whether it is intelligent to attack an opponent by comparing the total number of troops in our zone versus the opponent's zone.  In the event an attack is not advised, substantial processing time will be saved by reducing the instances of the UCT algorithm.  An additional modification we are contemplating, provided there is sufficient time in the semester, is the incorporation of a minimax approach to evaluate our moves and the opponent's moves.  Also, we may try prioritizing actions based on the probability of success, prioritizing certain events that may lead to more promising future moves that are consistent with our strategy, and different thresholds for limiting the depth in which UCT explores a game tree. The specific details of these modifications will be determined during our development and testing tasks and then formally documented in our final report.

In terms of evaluation, we will simulate several different approaches that incorporate as many modifications as time permits.  Each solution will be simulated in a large number of trials with a randomly generated map and then the performance will be assessed based on the number of games won as well as the rate at which the game was won.  In this case, the rate of winning a game is defined as the number of rounds required to win divided by the maximum number of rounds the game could have been played. The Warlight AI Challenge 2 also provides a leader board that indicates the users with the best performing algorithms. At a high-level we plan to compare the performance of our approach with the leaders in the challenge.

Hypothetically, the proposed approach could produce largely-varied results and is possibly dependent on the size of the map or the complexity of the opponent's algorithm. We seek to characterize the overall performance as it relates to these potential dependencies.  Our analysis and findings will be detailed in final report and presentation.

%Essential elements such as parsing routines as sensor/actuator platform will generate commands for the engine, events, create/update sate variables and possibly some book-keeping resources such as look-up tables (to be used by probabilistic routines) will also need to be developed. State variables, tasks and events will work as the  communication scheme between each level of this hierarchical adversarial planner.


%
\section{Project Management}\label{sec:management}
The main tasks for this project include implementing the UCT algorithm, conducting simulations, and evaluating the generated data so that we can publish our results in a report and a presentation. 

\subsection{Tasks}\label{sec:tasks}
There are seven main tasks required to successfully develop a planner that is capable of winning a game in the Warlight AI Challenge 2.  The first, which we have already completed, is a literature review of the UCT algorithm, its variants, and its application to the planning of games involving artificial intelligence. The next task is formulating the problem in a manner that is compatible with our proposed approach. With a concrete understanding for a possible solution, the next task is to implement the basic UCT algorithm.  The next two tasks consist of testing, modifying, and improving our implementation to enhance the performance of our solution.  Extensive simulation and evaluation of our algorithm is required to determine how well it performs compared to other strategies. Finally, we will present our results in a report and a presentation.  These tasks, with their respective tentative target dates are summarized in Table~\ref{tab:deadlines}.

%
\begin{table}[htbp]
  \centering
  \begin{tabular}{|c|c|}
    \hline
    \emph{Task} & \emph{Target Date} \\ 
    \hline
    Literature Review & March 9, 2015 \\ \hline
    Problem Formulation & March 16, 2015 \\ \hline
    UCT Implementation & April 6, 2015 \\ \hline
    Testing & April 13, 2015 \\ \hline
    UCT Modification and Improvement & April 13, 2015 \\ \hline
    Simulation and Evaluation & April 27, 2015 \\ \hline
    Write Report and Presentation & May 2, 2015 \\ \hline
  \end{tabular}
  \caption{Table of project tasks and target dates for completing each task.}
  \label{tab:deadlines}
\end{table}
%

\subsection{Assignments}\label{sec:assignments}
All three group members will be responsible for contributing to the problem formulation. This ensures that everyone in the group fully understands both the problem at hand and the proposed solution.  Jason Gregory and Kamruzzaman Quddus will work together to implement the proposed UCT algorithm and its modifications.  Moshe Katz will incrementally test the team's implementation as the code is being written to ensure correct functionality and then collect data by executing numerous simulations. The group, as a whole, will collaboratively evaluate the result, develop a thorough analysis, and create the paper and presentation.


%
%\begin{figure}[!htbp]
%  \centering
%  \includegraphics[width=0.95\columnwidth]{gantt_chart}
%  \caption{Tentative schedule for completing each task. Note, red boxes indicate deliverables that will be submitted to the professor.}
%  \label{fig:schedule}
%\end{figure}

%
\section{Conclusions}\label{sec:conclusions}
In this proposal we present our anticipated efforts for developing a probabilistic planner whose goal is to consistently win games in the Warlight AI Challenge 2. On the surface, this toy problem may appear to have little value as it seems inapplicable to real-world problems; however, upon further analysis one immediately sees the expected contributions from our technical approach.  Although UCT has been applied to choosing RISK territories \cite{riskalb}, to our knowledge this will be the first application and modification of the UCT algorithm in the context of the Warlight game. \newline
% References 
\bibliographystyle{plain}
\bibliography{references}

% End document
\end{document}
